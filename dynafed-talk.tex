\documentclass{beamer}
\usepackage[utf8]{inputenc}
\usepackage[T1]{fontenc}
\usepackage{hyperref}
\usepackage{graphicx}
\title{Update on the Dynamic Federation}
\date[ISPN ’80]{Technical Coordination Board}
\author[Euclid]{Frank Berghaus \href{mailto:berghaus@cern.ch}{\texttt{berghaus@cern.ch}}}

% --- Blocks ---
\setbeamertemplate{blocks}[default]

\usetheme{frank}

\begin{document}

\begin{frame}
\titlepage
\end{frame}


\begin{frame}
\frametitle{There Is No Largest Prime Number}
\framesubtitle{The proof uses \textit{reductio ad absurdum}.}
\begin{theorem}
There is no largest prime number. \end{theorem}
\begin{enumerate}
\item<1-| alert@1> Suppose $p$ were the largest prime number.
\item<2-> Let $q$ be the product of the first $p$ numbers.
\item<3-> Then $q+1$ is not divisible by any of them.
\item<1-> But $q + 1$ is greater than $1$, thus divisible by some prime
number not in the first $p$ numbers.
\end{enumerate}
\end{frame}

\begin{frame}
\frametitle{There Is No Largest Prime Number}
\framesubtitle{The proof uses \textit{reductio ad absurdum}.}
\begin{block}{Observation 1}
Simmons Hall is composed of metal and concrete.
\end{block}
\end{frame}

\end{document}
